\documentclass[11pt,a4paper,sans]{moderncv}        % possible options include font size ('10pt', '11pt' and '12pt'), paper size ('a4paper', 'letterpaper', 'a5paper', 'legalpaper', 'executivepaper' and 'landscape') and font family ('sans' and 'roman')

% moderncv themes
\moderncvstyle{banking}                             
% style options are 'casual' (default), 'classic', 'banking', 'oldstyle' and 'fancy'
\moderncvcolor{black}                               % color options 'black', 'blue' (default), 'burgundy', 'green', 'grey', 'orange', 'purple' and 'red'
%\renewcommand{\familydefault}{\sfdefault}         % to set the default font; use '\sfdefault' for the default sans serif font, '\rmdefault' for the default roman one, or any tex font name
%\nopagenumbers{}                                  % uncomment to suppress automatic page numbering for CVs longer than one page

% character encoding
\usepackage[utf8]{inputenc}                       % if you are not using xelatex ou lualatex, replace by the encoding you are using
%\usepackage{CJKutf8}                              % if you need to use CJK to typeset your resume in Chinese, Japanese or Korean

\usepackage{enumerate}
\usepackage{enumitem}

% adjust the page margins
\usepackage[scale=0.75, margin=0.6in]{geometry}
%\usepackage{etoolbox,xpatch}

%\usepackage{helvet} % Default font is the helvetica postscript font

%https://tex.stackexchange.com/questions/207374/page-numbers-with-moderncv-package-on-mactex-2014-dont-show
\usepackage{lastpage}

%\setlength{\hintscolumnwidth}{3cm}                % if you want to change the width of the column with the dates
%\setlength{\makecvtitlenamewidth}{10cm}           % for the 'classic' style, if you want to force the width allocated to your name and avoid line breaks. be careful though, the length is normally calculated to avoid any overlap with your personal info; use this at your own typographical risks...


%---ADD THE PHOTO---%
\patchcmd{\maketitle}
  {\hfil}
  {\hspace*{0.15\textwidth}}
  {}
  {}
\patchcmd{\maketitle}
  {\setlength{\maketitlewidth}{0.8\textwidth}}
  {\setlength{\maketitlewidth}{0.67\textwidth}}
  {}
  {}
\patchcmd{\maketitle}
  {\\[2.5em]}
  {\hfil\raisebox{-.7cm}{\framebox{\includegraphics[width=\@photowidth]{\@photo}}}\\[2.5em]}
  {}
  {}


% Code to calculate number of months for current job starts here
%\difftoday{2015}{07}{02} - this is base date (year-month-date)
%http://tex.stackexchange.com/questions/14518/difference-between-two-dates
\usepackage{datenumber}
\usepackage{calc}
\newcounter{datetoday}
\newcounter{diffyears}
\newcounter{diffmonths}
\newcounter{diffdays}

\newcommand{\difftoday}[3]{%
	\setmydatenumber{datetoday}{\the\year}{\the\month}{\the\day}%
	\setmydatenumber{diffdays}{#1}{#2}{#3}%
	\addtocounter{diffdays}{-\thedatetoday}%
	
	\ifnum\value{diffdays} > 0
		\def\diffbefore{}%
		\def\diffafter{}%
	\else
		\def\diffbefore{}%
		\def\diffafter{}%
		\setcounter{diffdays}{-\value{diffdays}}%
	\fi
	
	\setcounter{diffyears}{\value{diffdays}/365}%
	\setcounter{diffdays}{\value{diffdays}-365*\value{diffyears}}%
	\setcounter{diffmonths}{\value{diffdays}/30}%
	\setcounter{diffdays}{\value{diffdays}-30*\value{diffmonths}}%
	
	\diffbefore
		\ifnum\value{diffyears}=0
		\else
			\ifnum\value{diffyears} > 1
				\thediffyears\space years
			\else
				\thediffyears\space year
			\fi
		\fi
	
		\ifnum\value{diffmonths}=0
		\else
			\ifnum\value{diffmonths}>1
				\thediffmonths\space months\relax
			\else
				\thediffmonths\space month\relax
			\fi
		\fi
	\diffafter
 }

% Code to calculate number of months for current job ends here


\pagestyle{fancy}
\lhead{}
\chead{}
\rhead{}
\lfoot{\textcolor{gray}{\today \space| Mantosh}}
\cfoot{}
\rfoot{\textcolor{gray}{\thepage/\pageref{LastPage}}}

\recomputelengths                             % required when changes are made to page layout lengths

% personal data
%\name{MANTOSH}{KUMAR}
\firstname{Mantosh}
\familyname{Kumar}

%\title{Curriculum Vit\ae{}}                             % optional, remove / comment the line if not wanted

%\address{street and number}{postcode city}  % optional, remove / comment the line if not wanted; the "postcode city" and "country" arguments can be omitted or provided empty
\address{Hamilton}{Ontario}{Canada}

% optional, remove / comment the line if not wanted; the optional "type" of the phone can be "mobile" (default), "fixed" or "fax"

\phone[mobile]{+49~(160)~260~8158}

\email{mantoshk234@gmail.com}                               
%\homepage{www.johndoe.com}
\social[linkedin]{mantoshk}
%\social[twitter]{jdoe}
\social[github]{mantoshkumar1}

%\extrainfo{additional information} 

%\photo[64pt][0.4pt]{YOURPIC name without extension in jpg and eps format}                       % optional, remove / comment the line if not wanted; '64pt' is the height the picture must be resized to, 0.4pt is the thickness of the frame around it (put it to 0pt for no frame) and 'picture' is the name of the picture file

%\quote{Changing my world, one keystroke at a time!}        % optional, remove / comment the line if not wanted

% bibliography adjustements (only useful if you make citations in your resume, or print a list of publications using BibTeX)
%   to show numerical labels in the bibliography (default is to show no labels)
\makeatletter\renewcommand*{\bibliographyitemlabel}{\@biblabel{\arabic{enumiv}}}\makeatother
%   to redefine the bibliography heading string ("Publications")
%\renewcommand{\refname}{Articles}

% bibliography with mutiple entries
%\usepackage{multibib}
%\newcites{book,misc}{{Books},{Others}}

%------- content starts here --------------


\begin{document}
%\begin{CJK*}{UTF8}{gbsn}                          % to typeset your resume in Chinese using CJK
%-----       resume       ---------------------------------------------------------
\makecvtitle

\vspace{-38pt}
% OBJECTIVE SECTION
\section{\centerline{Summary}}
\begin{center}

	
Software Developer with over 14 years of experience across multiple phases of the software development lifecycle. Strong competence in C, Python, Django, system design, and back-end development. No sponsor required for employment in the US, Canada, or India.

\vspace{4pt}
\textbf{GitHub Profile}: {\color{blue}\href{https://github.com/mantoshkumar1}{https://github.com/mantoshkumar1}}
    
%\textbf{Work Permit}: Canada, India, Germany
    
\end{center} 

%fix: \normalsize is used because cventry has smaller font size and less vertical space than cvitem. \normalsize makes it equal to cvitem

%\section{Master thesis}
%\cvline{title}{\emph{Title}}
%\cvline{supervisors}{Supervisors}
%\cvline{description}{\small Short thesis abstract}

%\section{Computer skills}
%\cvcomputer{category 1}{XXX, YYY, ZZZ}{category 3}{XXX, YYY, ZZZ}
%\cvcomputer{category 2}{XXX, YYY, ZZZ}{category 4}{XXX, YYY, ZZZ}

%\vspace{-16pt}
\section{\centerline{Professional Experience}}
%\cventry{year--year}{Job title}{Employer}{City}{}{Description}  % arguments 3 to 6 are optional
%\cventry{year--year}{Job title}{Employer}{City}{}{Description line 1\newline{}Description line 2}% arguments 3 to 6 are optional
%\difftoday{2018}{5}{15}

\cventry{Aug 2020--Present}{Staff Test/Automation Engineer}{Nokia Corporation}{Ottawa, Canada}{}{
	\normalsize{
		\begin{itemize} \itemsep -5pt % -1 or -5pt
			\item{Designing and developing an automated testing framework for next-generation optical networking devices.}
			\item{Gathering test requirements directly from feature architects and formulating automatable test cases.}
			\item{Migrated legacy testing framework to the latest Infinera proprietary platform (PACE), fully integrated into CI/CD pipeline. Ensured no bugs and completed end-to-end in 5 months with 1000+ test scripts.}
			\item{Maintaining system integration across testing framework, CI/CD, and 24/7 test cycles.}
			\item{Mentoring team members, interviewing candidates for software dev positions, reviewing code, and collaborating with cross-functional teams and stakeholders.}
			\item{Technologies: Python, gRPC, YAML, Git}
		\end{itemize}
	}
}

\vspace{2pt}

\cventry{May 2018--Nov 2019 (1 year 6 months)}{Backend Engineer}{KI labs}{Munich, Germany}{}{
	\normalsize{
		\begin{itemize} \itemsep -5pt % -1 or -5pt
			\item{Built the MVP of a B2B chemical e-commerce platform — among the first in the European market.}
			\item {Developed RESTful APIs using Django; designed databases following n-tier architecture.}
			\item {Collaborated with front-end teams and product managers to deliver end-to-end features.}
			\item{Focused on code quality through unit tests and Cypress-based end-to-end automation.}
			\item{Automated CI/CD pipelines with Jenkins to streamline deployments.}
			\item  {Website: \color{blue}\href{https://chemondis.com/marketplace/}{https://chemondis.com/marketplace/}}
			\item {Technologies: Python, Django, PostgreSQL, RESTful, Cypress}
		\end{itemize}
	}
}

\vspace{2pt}

\cventry{Feb 2017--Sept 2017 (7 months)}{Research and Development Intern}{Siemens AG}{Munich, Germany}{}{
	\normalsize{
		\begin{itemize} \itemsep -5pt % -1 or -5pt
			\item {Designed a RESTful control interface and real-time monitoring system for SDN/NFV-based industrial networks.}
			\item{Developed a proof of concept for intent specification and monitoring for time-sensitive applications.}
			\item  {Website: \color{blue}\href{www.virtuwind.eu}{www.virtuwind.eu}}
			\item {Technologies: Java, Python, RESTCONF, SDN, Mininet, Yang, Maven, OSGi}
		\end{itemize}
	}
}

\vspace{2pt}

\cventry{Feb 2015--Feb 2017 (2 years)}{Protocol Stack Engineer (Part-time)}{Intel Corporation}{Munich, Germany}{}{	
	\normalsize{
		\begin{itemize} \itemsep -5pt % -1 or -5pt
			\item{Contributed to Radio Resource Control (RRC) development and testing for Intel’s XMM 7360 modem platform.}		
			\item{Produced functional specs and implemented 3GPP-aligned features with Intel’s internal requirements.}
			\item{Built Matlab tools to visualize OTA message traces and analyze modem performance metrics.}
			\item{Developed ENAS/UICC decoder modules and initiated automation framework for unit testing.}
			\item{Automated project status monitoring and collaborated closely with internal users for improvements.}
			\item {Languages: C++, Matlab, Python}
		\end{itemize}
	}
}

%\vspace{2pt} % ideally should have been 12 but due to formatting its becoming bad. Adjust it when you add a new position.
%\newpage

\cventry{Jan 2012--Sept 2014 (2 years 8 months)}{LTE Engineer}{Cisco Systems}{Bangalore, India}{}{
	\normalsize{
		\begin{itemize} \itemsep -5pt % -1 or -5pt
			\item{Contributed to the development of core network solutions for 2G/3G/LTE/Wi-Fi mobile networks.}
			\item{Built simulation tools to test and validate mobile network behavior, improving deployment efficiency.}
			\item{Collaborated with cross-functional teams and stakeholders to ensure successful delivery and support.}
			\item{Technologies: Python, Wireshark}
		\end{itemize}
		}
}


\vspace{2pt}

\cventry{Jan 2011--Jan 2012 (1 year)}{Software Engineer}{Aricent Group}{Bangalore, India}{}{
	\normalsize{
		\begin{itemize} \itemsep -5pt % -1 or -5pt
			\item{Built a prototype of a location-enabled social networking application showcased at Mobile World Congress 2012.}
			\item{Created use cases and process flows to support business requirements and integrated social and location-based features.}
			\item{Technologies: Java, Web2.0, MySQL}
		\end{itemize}
	}
}

\section{\centerline{Education}}
%\cventry{year--year}{Degree}{Institution}{City}{\textit{Grade}}{Description}  % arguments 3 to 6 are optional
\cventry{Oct 2014--Oct 2017}{M.Sc. in Computer Science (2.4/5, Best: 1)}{Technical University of Munich}{Munich, Germany}{}{}

\cventry{July 2006--June 2010}{Bachelors in Computer Science (7.7/10, Best: 10)}{West Bengal University of Technology}{Kolkata, India}{}{}

%\vspace{-16pt}
%\section{\centerline{Related Academic Projects}}
% Note: labelindent (space between left margin and bullet) and labelsep (space between bullet and text)
%\textbf{Challenges of Applying ISO 26262 in the Automotive Domain}
%\vspace{-6pt}
%\begin{itemize}[leftmargin=*,labelindent=5.6mm,labelsep=2.3mm] \itemsep -5pt % -4 or -5pt
%    \item{Authored a seminar paper over ISO 26262 implementation challenges. Probed the idea behind ISO 26262, its problems and their solutions by combining standards to reuse the artifacts. Performed critical assessment of ISO 26262 for future vehicles, requirements and technological advances adaption.}
%    \item{Proposed a prototype solution in KPIT Sparkle 2017 to address Security and Privacy concerns of future vehicles by developing and installing a monitoring and permission system in the In-Vehicle network similar to the iOS operating system.}
%\end{itemize}

%\textbf{System/Software Safety Analysis and Assurance of TCAS}
%\vspace{-6pt}
%\begin{itemize}[leftmargin=*,labelindent=5.6mm,labelsep=2.3mm] \itemsep -5pt % -4 or -5pt
%	\item {Performed an empirical comparative study of industrial safety analysis techniques (FTA, FMEA and STPA), as a part of joint scientific experiment supervised by the University of Stuttgart and TU Munich.}    
%	\item {Examined these techniques against \textit{Traffic Collision Avoidance System (TCAS)}.}
%\end{itemize}

\section{\centerline{Technical Skills}}
%\cvdoubleitem{category 1}{XXX, YYY, ZZZ}{category 3}{XXX, YYY, ZZZ}
%\cvdoubleitem{category 2}{XXX, YYY, ZZZ}{category 4}{XXX, YYY, ZZZ}

\cvitem{Programming}{C, C++, Java, Python, Matlab, Yang, Bash}
\cvitem{Software Skills}{Flask, Django, SQL, PostgreSQL, RESTful}
\cvitem{Tools}{Git, UML, Maven, Wireshark, LaTeX, Cypress}
\cvitem{Platform/API}{Linux, SuperMUC, Repast HPC}
\cvitem{Profiling}{Valgrind, Gprof, Perf, PAPI, VTune Amplifier XE, Scalasca}

\section{\centerline{Awards and Recognition}}
	\begin{itemize} %\itemsep -4pt
		\item {\cvitem{Heroes of Tomorrow Award, Intel Corporation (Jan \& Dec 2016)} 
			{Received twice in one year for outstanding contributions to the PS module of Intel XMM 7360 modem.}}
		\item {\cvitem{Graduate Aptitude Test in Engineering (GATE) - Computer Science}
		{National-level exam conducted by IISc and IITs for engineering graduates.}}
		\begin{itemize} \itemsep -1pt %-3 or -5pt
			\vspace{-11pt}
			\item \textbf{GATE 2013:} Top \textbf{0.87\%} (Out of 224160 applicants)
			\item \textbf{GATE 2012:} Top \textbf{0.76\%} (Out of 156780 applicants)
		\end{itemize}
		
	\end{itemize}

	
% \section{\centerline{Languages}}
%\textbf{Native:} Bhojpuri, Hindi \hfill \textbf{Full professional:} English \hfill \textbf{Elementary:} German (A1.1)
% \begin{center} English, Hindi, Bhojpuri \end{center}

%\section{Extra 1}
%\cvlistitem{Item 1}
%\cvlistitem{Item 2}This item is particularly long and therefore normally spans over several lines. Did you notice the indentation when the line wraps?}
%
%\section{Extra 2}
%\cvlistdoubleitem{Item 1}{Item 4}
%\cvlistdoubleitem{Item 2}{Item 5\cite{book1}}
%\cvlistdoubleitem{Item 3}{Item 6. Like item 3 in the single column list before, this item is particularly long to wrap over several lines.}
%+
%\section{References}
%\begin{cvcolumns}
%  \cvcolumn{Category 1}{\begin{itemize}\item Person 1\item Person 2\item Person 3\end{itemize}}
%  \cvcolumn{Category 2}{Amongst others:\begin{itemize}\item Person 1, and\item Person 2\end{itemize}(more upon request)}
%  \cvcolumn[0.5]{All the rest \& some more}{\textit{That} person, and \textbf{those} also (all available upon request).}
%\end{cvcolumns}

% Publications from a BibTeX file without multibib
%  for numerical labels: \renewcommand{\bibliographyitemlabel}{\@biblabel{\arabic{enumiv}}}% CONSIDER MERGING WITH PREAMBLE PART
%  to redefine the heading string ("Publications"): \renewcommand{\refname}{Articles}
\nocite{*}
\bibliographystyle{plain}
%\bibliography{publications}                        % 'publications' is the name of a BibTeX file

% Publications from a BibTeX file using the multibib package
%\section{Publications}
%\nocitebook{book1,book2}
%\bibliographystylebook{plain}
%\bibliographybook{publications}                   % 'publications' is the name of a BibTeX file
%\nocitemisc{misc1,misc2,misc3}
%\bibliographystylemisc{plain}
%\bibliographymisc{publications}                   % 'publications' is the name of a BibTeX file

%\clearpage
%\clearpage\end{CJK*}                              % if you are typesetting your resume in Chinese using CJK; the \clearpage is required for fancyhdr to work correctly with CJK, though it kills the page numbering by making \lastpage undefined
\end{document}


%% end of file `template.tex'.
