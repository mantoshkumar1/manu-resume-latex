\documentclass[11pt,a4paper,sans]{moderncv}        % possible options include font size ('10pt', '11pt' and '12pt'), paper size ('a4paper', 'letterpaper', 'a5paper', 'legalpaper', 'executivepaper' and 'landscape') and font family ('sans' and 'roman')

% moderncv themes
\moderncvstyle{banking}                             
% style options are 'casual' (default), 'classic', 'banking', 'oldstyle' and 'fancy'
\moderncvcolor{black}                               % color options 'black', 'blue' (default), 'burgundy', 'green', 'grey', 'orange', 'purple' and 'red'
%\renewcommand{\familydefault}{\sfdefault}         % to set the default font; use '\sfdefault' for the default sans serif font, '\rmdefault' for the default roman one, or any tex font name
%\nopagenumbers{}                                  % uncomment to suppress automatic page numbering for CVs longer than one page

% character encoding
\usepackage[utf8]{inputenc}                       % if you are not using xelatex ou lualatex, replace by the encoding you are using
%\usepackage{CJKutf8}                              % if you need to use CJK to typeset your resume in Chinese, Japanese or Korean

\usepackage{enumerate}
\usepackage{enumitem}

% adjust the page margins
\usepackage[scale=0.75, margin=0.6in]{geometry}
%\usepackage{etoolbox,xpatch}

%\usepackage{helvet} % Default font is the helvetica postscript font

%https://tex.stackexchange.com/questions/207374/page-numbers-with-moderncv-package-on-mactex-2014-dont-show
\usepackage{lastpage}

%\setlength{\hintscolumnwidth}{3cm}                % if you want to change the width of the column with the dates
%\setlength{\makecvtitlenamewidth}{10cm}           % for the 'classic' style, if you want to force the width allocated to your name and avoid line breaks. be careful though, the length is normally calculated to avoid any overlap with your personal info; use this at your own typographical risks...


%---ADD THE PHOTO---%
\patchcmd{\maketitle}
  {\hfil}
  {\hspace*{0.15\textwidth}}
  {}
  {}
\patchcmd{\maketitle}
  {\setlength{\maketitlewidth}{0.8\textwidth}}
  {\setlength{\maketitlewidth}{0.67\textwidth}}
  {}
  {}
\patchcmd{\maketitle}
  {\\[2.5em]}
  {\hfil\raisebox{-.7cm}{\framebox{\includegraphics[width=\@photowidth]{\@photo}}}\\[2.5em]}
  {}
  {}


% code for skype icon starts here
% reference: https://tex.stackexchange.com/questions/190954/skype-icon-on-moderncv?noredirect=1&lq=1
\usepackage{tikz}
\newcommand*{\skypesocialsymbol} {%
	\protect\raisebox{-0.085em}{%
		\protect\begin{tikzpicture}[y=0.08em,x=0.08em,xscale=0.022,yscale=-0.022, inner sep=0pt, outer sep=0pt]
		\protect\path[fill=color2,even odd rule] (487.6550,288.9690) .. controls (489.0610,278.5690) and
		(489.8700,267.9960) .. (489.8700,257.2330) .. controls (489.8700,128.0770) and
		(384.5990,23.3610) .. (254.7670,23.3610) .. controls (241.8630,23.3610) and
		(229.2120,24.4210) .. (216.9010,26.4410) .. controls (194.8280,12.0570) and
		(168.5590,3.6740) .. (140.2880,3.6740) .. controls (62.7660,3.6740) and
		(0.0000,66.4820) .. (0.0000,143.9800) .. controls (0.0000,172.1780) and
		(8.2990,198.3740) .. (22.5900,220.3690) .. controls (20.6650,232.3860) and
		(19.6810,244.6920) .. (19.6810,257.2290) .. controls (19.6810,386.4050) and
		(124.8980,491.1100) .. (254.7660,491.1100) .. controls (269.4230,491.1100) and
		(283.6930,489.6840) .. (297.5620,487.1780) .. controls (319.1120,500.5470) and
		(344.4960,508.3260) .. (371.7080,508.3260) .. controls (449.2100,508.3260) and
		(512.0010,445.5020) .. (512.0010,368.0120) .. controls (511.9980,338.7190) and
		(503.0410,311.4840) .. (487.6550,288.9690) -- cycle(276.7400,429.5960) ..
		controls (202.0340,433.4870) and (167.0750,416.9590) .. (135.0500,386.9050) ..
		controls (99.2850,353.3370) and (113.6520,315.0500) .. (142.7900,313.1040) ..
		controls (171.9120,311.1590) and (189.3980,346.1160) .. (204.9410,355.8400) ..
		controls (220.4650,365.5280) and (279.5340,387.6000) .. (310.7350,351.9320) ..
		controls (344.7100,313.1040) and (288.1410,293.0120) .. (246.6760,286.9300) ..
		controls (187.4730,278.1640) and (112.7260,246.1370) .. (118.5410,183.0230) ..
		controls (124.3580,119.9490) and (172.1230,87.6090) .. (222.3910,83.0470) ..
		controls (286.4680,77.2300) and (328.1820,92.7540) .. (361.1760,120.9070) ..
		controls (399.3270,153.4360) and (378.6840,189.8010) .. (354.3770,192.7270) ..
		controls (330.1660,195.6360) and (302.9730,139.2230) .. (249.5860,138.3750) ..
		controls (194.5590,137.5110) and (157.3690,195.6360) .. (225.3000,212.1590) ..
		controls (293.2660,228.6640) and (366.0500,235.4450) .. (392.2610,297.5760) ..
		controls (418.4900,359.7130) and (351.5070,425.7010) .. (276.7400,429.5960) --
		cycle;
		\protect\end{tikzpicture}}%
	~}
% code for skype icon ends here

% Code to calculate number of months for current job starts here
%\difftoday{2015}{07}{02} - this is base date (year-month-date)
%http://tex.stackexchange.com/questions/14518/difference-between-two-dates
\usepackage{datenumber}
\usepackage{calc}
\newcounter{datetoday}
\newcounter{diffyears}
\newcounter{diffmonths}
\newcounter{diffdays}

\newcommand{\difftoday}[3]{%
	\setmydatenumber{datetoday}{\the\year}{\the\month}{\the\day}%
	\setmydatenumber{diffdays}{#1}{#2}{#3}%
	\addtocounter{diffdays}{-\thedatetoday}%
	
	\ifnum\value{diffdays} > 0
		\def\diffbefore{}%
		\def\diffafter{}%
	\else
		\def\diffbefore{}%
		\def\diffafter{}%
		\setcounter{diffdays}{-\value{diffdays}}%
	\fi
	
	\setcounter{diffyears}{\value{diffdays}/365}%
	\setcounter{diffdays}{\value{diffdays}-365*\value{diffyears}}%
	\setcounter{diffmonths}{\value{diffdays}/30}%
	\setcounter{diffdays}{\value{diffdays}-30*\value{diffmonths}}%
	
	\diffbefore
		\ifnum\value{diffyears}=0
		\else
			\ifnum\value{diffyears} > 1
				\thediffyears\space years
			\else
				\thediffyears\space year
			\fi
		\fi
	
		\ifnum\value{diffmonths}=0
		\else
			\ifnum\value{diffmonths}>1
				\thediffmonths\space months\relax
			\else
				\thediffmonths\space month\relax
			\fi
		\fi
	\diffafter
 }

% Code to calculate number of months for current job ends here


\pagestyle{fancy}
\lhead{}
\chead{}
\rhead{}
\lfoot{\textcolor{gray}{\today \space| Mantosh}}
\cfoot{}
\rfoot{\textcolor{gray}{\thepage/\pageref{LastPage}}}

\recomputelengths                             % required when changes are made to page layout lengths

% personal data
%\name{MANTOSH}{KUMAR}
\firstname{Mantosh}
\familyname{Kumar}

%\title{Curriculum Vit\ae{}}                             % optional, remove / comment the line if not wanted

%\address{street and number}{postcode city}  % optional, remove / comment the line if not wanted; the "postcode city" and "country" arguments can be omitted or provided empty
\address{Hamilton}{Ontario}{Canada}

% optional, remove / comment the line if not wanted; the optional "type" of the phone can be "mobile" (default), "fixed" or "fax"
\phone[mobile]{+1~(647)~609~5999}
%\phone[mobile]{+49~(160)~260~8158}

\email{mantoshk234@gmail.com}                               
%\homepage{www.johndoe.com}
\social[linkedin]{mantoshk}
%\social[twitter]{jdoe}
\social[github]{mantoshkumar1}

\extrainfo{\skypesocialsymbol~\href{https://web.skype.com/en/}{live:mantoshk234}}


%\extrainfo{additional information} 

%\photo[64pt][0.4pt]{YOURPIC name without extension in jpg and eps format}                       % optional, remove / comment the line if not wanted; '64pt' is the height the picture must be resized to, 0.4pt is the thickness of the frame around it (put it to 0pt for no frame) and 'picture' is the name of the picture file

%\quote{Changing my world, one keystroke at a time!}        % optional, remove / comment the line if not wanted

% bibliography adjustements (only useful if you make citations in your resume, or print a list of publications using BibTeX)
%   to show numerical labels in the bibliography (default is to show no labels)
\makeatletter\renewcommand*{\bibliographyitemlabel}{\@biblabel{\arabic{enumiv}}}\makeatother
%   to redefine the bibliography heading string ("Publications")
%\renewcommand{\refname}{Articles}

% bibliography with mutiple entries
%\usepackage{multibib}
%\newcites{book,misc}{{Books},{Others}}

%------- content starts here --------------


\begin{document}
%\begin{CJK*}{UTF8}{gbsn}                          % to typeset your resume in Chinese using CJK
%-----       resume       ---------------------------------------------------------
\makecvtitle

\vspace{-38pt}
% OBJECTIVE SECTION
\section{\centerline{Summary}}
\begin{center}

	
Software Developer (Computer Science, M.Sc.) with over 8 years of experience in multiple phases of software development lifecycle. Strong competence in C, Python, Django, System design and Back-end development.%3GPP.

%Proficient with C, Python, Django, System design and 3GPP.

\vspace{4pt}
\textbf{GitHub Profile}: {\color{blue}\href{https://github.com/mantoshkumar1}{https://github.com/mantoshkumar1}}
    
%\textbf{Work Permit}: Canada, India, Germany
    
\end{center} 

%fix: \normalsize is used because cventry has smaller font size and less vertical space than cvitem. \normalsize makes it equal to cvitem

%\section{Master thesis}
%\cvline{title}{\emph{Title}}
%\cvline{supervisors}{Supervisors}
%\cvline{description}{\small Short thesis abstract}

%\section{Computer skills}
%\cvcomputer{category 1}{XXX, YYY, ZZZ}{category 3}{XXX, YYY, ZZZ}
%\cvcomputer{category 2}{XXX, YYY, ZZZ}{category 4}{XXX, YYY, ZZZ}

%\vspace{-16pt}
\section{\centerline{Professional Experience}}
%\cventry{year--year}{Job title}{Employer}{City}{}{Description}  % arguments 3 to 6 are optional
%\cventry{year--year}{Job title}{Employer}{City}{}{Description line 1\newline{}Description line 2}% arguments 3 to 6 are optional
%\difftoday{2018}{5}{15}

\cventry{May 2018--Nov 2019 (1 year 6 months)}{Backend Software Engineer}{KI labs}{Munich, Germany}{}{
	\normalsize{
		\begin{itemize} \itemsep -5pt % -1 or -5pt
			\item{Built the MVP (Minimum Viable Product) version of B2B chemical e-commerce platform - First of its kind in European chemical digital marketplace.}
			\item {Designed and developed server-side RESTful services and API using Django implementation.}
			\item {Designed databases and table structures following n-tier architecture methodology for web applications.}
			\item{Collaborated with front-end engineers to see the project through, from conception to completion and integrated user-facing elements developed by front-end developers with server-side logic.}
			\item{Worked closely with product managers and business partners to understand software requirements.}
			\item{Developed unit test framework to test the internal structure of the application and ensure application quality.}
			\item{Converted manual test cases to automated end-to-end flow functionalities.}
			%\item{Automated build and deployment using Jenkins to reduce human error and speed up production processes.}
			\item  {Website: \color{blue}\href{https://chemondis.com/marketplace/}{https://chemondis.com/marketplace/}}
			\item {Technologies: Python, Django, PostgreSQL, RESTful, Cypress}
		\end{itemize}
	}
}

\vspace{2pt}

\cventry{Feb 2017--Sept 2017 (7 months)}{Research and Development Intern}{Siemens AG}{Munich, Germany}{}{
	\normalsize{
		\begin{itemize} \itemsep -5pt % -1 or -5pt
			\item {Designed an intent-based RESTful control interface and real-time intent monitoring system for multi-tenant SDN and NFV based industrial network.}
			\item{Built a proof of concept for the realization of intent specification and intent monitoring models for real-time sensitive applications.}
			\item  {Website: \color{blue}\href{www.virtuwind.eu}{www.virtuwind.eu}}
			\item {Technologies: Java, Python, RESTCONF, SDN, Mininet, Yang, Maven, OSGi}
		\end{itemize}
	}
}

\vspace{2pt}

\cventry{Feb 2015--Feb 2017 (2 years)}{Protocol Stack Engineer (Part-time)}{Intel Corporation}{Munich, Germany}{}{	
	\normalsize{
		\begin{itemize} \itemsep -5pt % -1 or -5pt
			\item{Worked on Radio Resource Control development activities for Intel proprietary platforms, including design implementation and host-based testing to enhance Intel modem XMM 7360.}
			\item{Produced functional specifications for the PS module of Intel modem (XMM 7360) and consolidated 3GPP specifications with Intel’s own internal requirements into software implementation.}
			\item {Produced Matlab resources to visualize Intel modem (XMM 7360) performance metrics and identify gaps for improvement using OTA (Over The Air) message traces.}
			\item {Developed modules of ENAS messages and UICC commands/files for Intel proprietary decoder.}
			\item{Worked on development of the initial version of unit-level test automation framework for UICC decoder module.}
			\item{Constantly communicated with users of Intel proprietary decoder and took note of requirements and features that can be added to the system.}
			\item{Fully automated the tasks of project content control and propagation status monitoring within the team.}
			\item {Languages: C++, Matlab, Python}
		\end{itemize}
	}
}

%\vspace{2pt} % ideally should have been 12 but due to formatting its becoming bad. Adjust it when you add a new position.
\newpage

\cventry{Jan 2012--Sept 2014 (2 years 8 months)}{LTE Engineer}{Cisco Systems}{Bangalore, India}{}{
	\normalsize{
		\begin{itemize} \itemsep -5pt % -1 or -5pt
			\item{Designed and developed a validation and deployment tool for ASR 5X00 Cisco Packet Core products.}
			\item {Developed a scalable functional \textit{Packet Core Network Simulation Environment} for 2G/3G/LTE/Wi-Fi and EHRPD network elements adhering to 3GPP specifications. Simulated GnGP, S1U, S3 and S4 interfaces.}
			\item{Worked on simulation of various mobile networking protocols such as PMIPv6, GTP, RANAP, NAS and S1AP.}
			\item{Worked on previously existing bottlenecks of the system and significantly decreased system initialization time.}
			\item{Conducted walk-through, requirements gathering meetings and worked with the stakeholders to ensure the right solution and provided comprehensive support to end-users and stakeholders through training and documentation.}
			\item {Technologies: Python, Wireshark}
		\end{itemize}
		}
}


\vspace{2pt}

\cventry{Jan 2011--Jan 2012 (1 year)}{Software Engineer}{Aricent Group}{Bangalore, India}{}{
	\normalsize{
		\begin{itemize} \itemsep -5pt % -1 or -5pt
			\item{Developed a \textit{Location-Enabled Social Network Prototype Application} for Mobile World Congress 2012.}
			\item {Developed detailed use cases and process flow diagrams to support functional specifications and integrated Facebook and location-based services on Web2.0 framework.}
			\item {Technologies: Java, HTML, CSS, JBPM, XML, JSON, JDBC, MySQL}
		\end{itemize}
	}
}

\section{\centerline{Education}}
%\cventry{year--year}{Degree}{Institution}{City}{\textit{Grade}}{Description}  % arguments 3 to 6 are optional
\cventry{Oct 2014--Oct 2017}{M.Sc. in Computer Science (2.4/5, Best: 1)}{Technical University of Munich}{Munich, Germany}{}{}

\cventry{July 2006--June 2010}{Bachelors in Computer Science (7.7/10, Best: 10)}{West Bengal University of Technology}{Kolkata, India}{}{}

%\vspace{-16pt}
%\section{\centerline{Related Academic Projects}}
% Note: labelindent (space between left margin and bullet) and labelsep (space between bullet and text)
%\textbf{Challenges of Applying ISO 26262 in the Automotive Domain}
%\vspace{-6pt}
%\begin{itemize}[leftmargin=*,labelindent=5.6mm,labelsep=2.3mm] \itemsep -5pt % -4 or -5pt
%    \item{Authored a seminar paper over ISO 26262 implementation challenges. Probed the idea behind ISO 26262, its problems and their solutions by combining standards to reuse the artifacts. Performed critical assessment of ISO 26262 for future vehicles, requirements and technological advances adaption.}
%    \item{Proposed a prototype solution in KPIT Sparkle 2017 to address Security and Privacy concerns of future vehicles by developing and installing a monitoring and permission system in the In-Vehicle network similar to the iOS operating system.}
%\end{itemize}

%\textbf{System/Software Safety Analysis and Assurance of TCAS}
%\vspace{-6pt}
%\begin{itemize}[leftmargin=*,labelindent=5.6mm,labelsep=2.3mm] \itemsep -5pt % -4 or -5pt
%	\item {Performed an empirical comparative study of industrial safety analysis techniques (FTA, FMEA and STPA), as a part of joint scientific experiment supervised by the University of Stuttgart and TU Munich.}    
%	\item {Examined these techniques against \textit{Traffic Collision Avoidance System (TCAS)}.}
%\end{itemize}

\section{\centerline{Technical Skills}}
%\cvdoubleitem{category 1}{XXX, YYY, ZZZ}{category 3}{XXX, YYY, ZZZ}
%\cvdoubleitem{category 2}{XXX, YYY, ZZZ}{category 4}{XXX, YYY, ZZZ}

\cvitem{Programming}{C, C++, Java, Python, Matlab, Yang, Bash}
\cvitem{Software Skills}{Flask, Django, SQL, PostgreSQL, RESTful}
\cvitem{Tools}{Git, UML, Maven, Wireshark, LaTeX, Cypress}
\cvitem{Platform/API}{Linux, SuperMUC, Repast HPC}
\cvitem{Profiling}{Valgrind, Gprof, Perf, PAPI, VTune Amplifier XE, Scalasca}

\section{\centerline{Awards and Recognition}}
	\begin{itemize} %\itemsep -4pt
		\item {\cvitem{Heroes of Tomorrow Award (Intel Corporation), January | December 2016} 
			{Awarded twice for outstanding contribution to PS module of Intel XMM 7360 modem.}}
		\item {\cvitem{Graduate Aptitude Test in Engineering (GATE) - Computer Science}
		{An all-India examination organized by the IISC and IITs on behalf of Government of India.}}
		\begin{itemize} \itemsep -1pt %-3 or -5pt
			\vspace{-11pt}
			\item \textbf{GATE 2013:} Among the top \textbf{0.87\%} (Out of 224160 applicants)
			\item \textbf{GATE 2012:} Among the top \textbf{0.76\%} (Out of 156780 applicants)
		\end{itemize}
		
	\end{itemize}
	
\section{\centerline{Languages}}
%\textbf{Native:} Bhojpuri, Hindi \hfill \textbf{Full professional:} English \hfill \textbf{Elementary:} German (A1.1)
\begin{center} English, Hindi, Bhojpuri \end{center}

%\section{Extra 1}
%\cvlistitem{Item 1}
%\cvlistitem{Item 2}This item is particularly long and therefore normally spans over several lines. Did you notice the indentation when the line wraps?}
%
%\section{Extra 2}
%\cvlistdoubleitem{Item 1}{Item 4}
%\cvlistdoubleitem{Item 2}{Item 5\cite{book1}}
%\cvlistdoubleitem{Item 3}{Item 6. Like item 3 in the single column list before, this item is particularly long to wrap over several lines.}
%+
%\section{References}
%\begin{cvcolumns}
%  \cvcolumn{Category 1}{\begin{itemize}\item Person 1\item Person 2\item Person 3\end{itemize}}
%  \cvcolumn{Category 2}{Amongst others:\begin{itemize}\item Person 1, and\item Person 2\end{itemize}(more upon request)}
%  \cvcolumn[0.5]{All the rest \& some more}{\textit{That} person, and \textbf{those} also (all available upon request).}
%\end{cvcolumns}

% Publications from a BibTeX file without multibib
%  for numerical labels: \renewcommand{\bibliographyitemlabel}{\@biblabel{\arabic{enumiv}}}% CONSIDER MERGING WITH PREAMBLE PART
%  to redefine the heading string ("Publications"): \renewcommand{\refname}{Articles}
\nocite{*}
\bibliographystyle{plain}
%\bibliography{publications}                        % 'publications' is the name of a BibTeX file

% Publications from a BibTeX file using the multibib package
%\section{Publications}
%\nocitebook{book1,book2}
%\bibliographystylebook{plain}
%\bibliographybook{publications}                   % 'publications' is the name of a BibTeX file
%\nocitemisc{misc1,misc2,misc3}
%\bibliographystylemisc{plain}
%\bibliographymisc{publications}                   % 'publications' is the name of a BibTeX file

%\clearpage
%\clearpage\end{CJK*}                              % if you are typesetting your resume in Chinese using CJK; the \clearpage is required for fancyhdr to work correctly with CJK, though it kills the page numbering by making \lastpage undefined
\end{document}


%% end of file `template.tex'.
